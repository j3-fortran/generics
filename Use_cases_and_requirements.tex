\documentclass{article}
\usepackage[utf8]{inputenc}
\usepackage{listings}

\newcounter{usecase}
\newcounter{minicase}
\newcounter{requirement}

%% Arguments: label, vote
\newcommand{\newusecase}[2]{
\refstepcounter{usecase}\label{uc:#1}
\subsubsection{Use case \ref{uc:#1}: #1 (#2)}}

\newcommand{\newrequirement}[2]{
\refstepcounter{requirement}\label{req:#1}
\paragraph{R\ref{req:#1}:} #2}

\title{Generics: Use Cases and Requirements}
\author{Generics subgroup}
\date{November 2020}

\begin{document}

\maketitle
References: 18-110r1

\section{Introduction}
\subsection{Organization of document}

\section{Use cases}

\subsection{Generic Algorithms}

A common justification for providing generics/templates for Fortran is
that they would enable the development of a library of algorithms that
would simplify the development of code.
Typically an algorithm can be implemented as a single procedure.
As generics, they could be implemented as one
procedure per generic module, multiple procedures per generic
module, or as individual generic procedures.
Their usage would be simplified if the generics allowed template
parameters to be applied to a procedure invocation as opposed to at
the type definition or module declaration level.  (ref. M. Haveraaen)
It appears that, for the algorithmic use cases below, the template
parameters can be inferred from the arguments to the procedure,
further simplifying their usage. As a result, while  not a requirement
for a generic algorithm, almost all generic algorithms would benefit
from   the  capabilities:
\begin{itemize}
\item the ability to define and instantiate a single generic procedure;

\item the ability to use type inference in instantiating a single
  generic procedure; and

\item instantiation in the same specification part that defines the
  data type.
\end{itemize}

The number of algorithms of potential interest is large.
For example, the C++ 20 Library defines about 200
algorithms in fourteen categories:
\begin{itemize}
\item Non-modifying sequence operations,
\item Modifying sequence operations,
\item Partitioning operations,
\item Sorting operations,
\item Binary search operations (on sorted ranges),
\item Other operations on sorted ranges,
\item Set operations (on sorted ranges),
\item Heap operations,
\item Minimum/maximum operations,
\item Comparison operations,
\item Permutation operations,
\item Numeric operations,
\item Operations on uninitialized memory, and
\item the C library.
\end{itemize}
Not all of these are of interest to a Fortran ``template'' library, i.e.,
``Heap operations'' and ``Operations on uninitialized  memory'' appear
to be of little interest to the Fortran community, and the ``C library''
consists of two non-templated algorithms. Further, many of the
algorithms are near duplicates of one another, differing only in
which form of iterator is used: the old style iterator or the newer
ranges. Still it appears that
well over 50 algorithms are of likely interest. To illustrate the
capabilities of generic algorithms we have tried to select one
representative algorithm from each of the eleven categories of
interest. 

\newusecase{Extending intrinsic algorithms}{0-0-0}
A somewhat common pattern is to desire to reproduce the algorithm of
an intrinsic procedure in the context of a custom derived type.
{\rm FINDLOC} is perhaps the most frequently cited example.  Here, the
user wishes to find the first entry in an array that equals a given
value.  The algorithm itself is the same for any type that 
supports tests for equality $==$, and can be generalized to any
container that supports iterators.

The generic {\rm FINDLOC} procedure has the following requirements:
\begin{itemize}
\item the ability to have a general type as a generic parameter
  either as an explicit parameter or as implicit in the iterators;

\item the ability to associate an equality operation with a generic
  type parameter, either implicitly
  as a property of the type definition, or as an explicit parameter to
  the generic with a defined interface;

\item the ability to specialize for different ranks of arrays; and

\item the ability to take one or more iterators as instantiation
  parameters.

\end{itemize}

\newusecase{Swap}{0-0-0}

A common task in code is to swap the values or targets of two
variables. The swap can involve different forms of variables:
\begin{itemize}
\item Swap non-polymorphic (TYPE) scalars,
\item Swap polymorphic (CLASS) scalars,
\item Swap scalars with LEN parameters,
\item Swap pointers,
\item Swap allocatables (via {\rm MOVE\_ALLOC}),
\item Swap arrays of the same shape, and
\item Swap ranges of elements of two containers.
\end{itemize}
Ideally the generics facility could provide means to allow a single template to work for all these variants.   The variants may require additional parameters to indicate the arguments's attributes.

The generic swap procedure has the following requirements:
\begin{itemize}
\item the ability to have a general type as a generic parameter;

\item the ability to associate an assignment operation with a type
  parameter, either implicitly as a property of all types, implicitly
  as a property of the type definition, or as an explicit parameter to
  the generic with a defined interface; and

\item the ability to specialize depending on whether
  \begin{itemize}
    \item an instantiation type parameter is polymorphic or not;
    \item an instantiation type has a LEN parameter or not;
    \item the arguments are pointers;
    \item the arguments are allocatables; or
    \item the  arguments are arrays.
  \end{itemize}

\end{itemize}

\newusecase{Partition}{0-0-0}
It is sometimes useful to divide a collection of items into two sets
selected according to the results of a logical predicate.  Such a
division is termed a partition. The partition procedure need to be
able to iterate through the collection, apply the predicate, sort the
results into two parts, and return a marker for the start of the
second   portion of the partition.

A generic partition procedure has the following requirements:
\begin{itemize}
\item the ability to have a general type as a generic parameter
  either as an explicit parameter or as implicit in the  iterators;

\item the ability to associate an assignment operation with a type
  parameter, either implicitly as a property of all types, or
  as a property of the type definition, or as an explicit parameter to
  the generic with a defined interface;

\item the ability to associate a unary logical predicate with a
  generic type parameter, either implicitly
  as a property of the type definition, or as an explicit parameter to
  the generic with a defined interface; and

\item the ability to take one or more iterators as instantiation
  parameters.

\end{itemize}

\newusecase{Sorting}{0-0-0}

While the need to sort a list of items is somewhat rarer in typical
Fortran applications than in wider software communities, it
nonetheless arises often enough to be a problem.   It is typically
applied to a rank one array, but can be generalized to any container
with appropriate iterators, e.g. Lists and Vectors.  A given sorting
algorithm can generally be applied to any type that provides a
comparison (`$<$' or `$>$') operation, or equivalent binary logical
predicate.   With current Fortran
capabilities, the algorithm must be re-implemented for each type -
with some possible simplification through the use of include files.

A generic sorting procedure has the following requirements:
\begin{itemize}
\item the ability to have a general type as a generic parameter
  either as an explicit parameter or as implicit in the iterators;

\item the ability to associate an assignment operation with a type
  parameter, either implicitly as a property of all types, or
  as a property of the type definition, or as an explicit parameter to
  the generic with a defined interface;

\item the ability to associate a comparison operation with a generic
  type parameter, either implicitly
  as a property of the type definition, or as an explicit parameter to
  the generic with a defined interface; and

\item the ability to take one or more iterators as instantiation
  parameters.

\end{itemize}

\newusecase{Searching}{0-0-0}

Searching is the  complement of sorting. One typically sorts an array
so that it can be searched for specific elements. A binary search
algorithm allows the searching of a sorted array with O(ln(n)) cost.
For the search to be efficient, it must use the same comparison
operation used in the sorting,  With current Fortran
capabilities, the algorithm must be re-implemented for each type -
with some possible simplification through the use of include
files. The requirements for the generic search are essentially the
same as for the corresponding sort algorithm.

\newusecase{Merging}{0-0-0}
It is sometimes useful to merge two sorted collections into one
sorted collection. The {\rm MERGE} procedure must take two
collections sorted using the same comparison operation, iterate
through both of them in the same direction comparing current elements
using the same comparison, and enter them into a third collection
according to the results of the comparison.

A generic {\rm MERGE} procedure has the following
requirements:
\begin{itemize}
\item the ability to have a general type as a generic parameter
  either as an explicit parameter or as implicit in the iterators;

\item the ability to associate an assignment operation with a type
  parameter, either implicitly as a property of all types, or
  as a property of the type definition, or as an explicit parameter to
  the generic with a defined interface;

\item the ability to associate a comparison operation with a generic
  type parameter, either implicitly
  as a property of the type definition, or as an explicit parameter to
  the generic with a defined interface; and

\item the ability to take two iterators as instantiation
  parameters.

\end{itemize}

\newusecase{Set intersection}{0-0-0}
A sorted container has properties similar to a set, and can easily be
converted to a set by removing adjacent duplicate values.  As a result
it is useful to define set operations, such as {\rm SET\_INTERSECTION},
for pairs of sorted containers.  In {\rm SET\_INTERSECTION}, the two
collections are scanned in sequence and for each value with m
copies in the first set and n copies in the second {\tt MIN(M,N)} copies
are copied to the container holding the intersection.

A generic {\rm SET\_INTERSECTION} procedure has the following
requirements:
\begin{itemize}
\item the ability to have a general type as a generic parameter
  either as an explicit parameter or as implicit in the iterators;

\item the ability to associate an assignment operation with a type
  parameter, either implicitly as a property of all types, or
  as a property of the type definition, or as an explicit parameter to
  the generic with a defined interface;

\item the ability to associate a comparison operation with a generic
  type parameter, either implicitly
  as a property of the type definition, or as an explicit parameter to
  the generic with a defined interface; and

\item the ability to take two iterators as instantiation
  parameters.

\end{itemize}

\newusecase{Max element}{0-0-0}
It  is often useful to know the maximum or minimum elements of a
collection. The function {\rm MAX\_ELEMENT} would return the maximum
element of a collection according to a comparison operation assumed to
be a less than operation, ``$<$'', perhaps supplemented by an equality
operation ``$==$'' to deal with {\rm NaN}s.  In {\rm MAX\_ELEMENT},
the elements of a collection are scanned in sequence, using an
iterator, replacing the current maximum element with the new one if
the new one compares false when the first argument in a comparison
with the current maximum, and true when  the second.

A generic {\rm MAX\_ELEMENT} procedure has the following
requirements:
\begin{itemize}
\item the ability to have a general type as a generic parameter
  either as an explicit parameter or as implicit in the iterator;

\item the ability to associate an assignment operation with a type
  parameter, either implicitly as a property of all types, or
  as a property of the type definition, or as an explicit parameter to
  the generic with a defined interface;

\item the ability to associate a comparison operation with a
  generic type parameter, either implicitly
  as a property of the type definition, or as an explicit parameter to
  the generic with a defined interface; and

\item the ability to take an iterator as an instantiation
  parameter.

\end{itemize}

\newusecase{Lexicographical Comparison}{0-0-0}
It is sometimes useful to perform a lexicographical comparison of two
collections using a binary logical predicate. The comparison  should
return -1 if the first collection is less than the second, 1 if the
first collection is greater, and 0 if they are equal. The collections
should be compared element by  element, with the first mismatching
element determining which is collection is less or greater than the
other, otherwise the shorter range is less than the other, otherwise
if the collections are the same length and the elements are
equivalent, the collections are considered equal.

A generic {\rm LEXICOGRAPHICAL\_COMPARISON} procedure has the
following requirements:
\begin{itemize}
\item the ability to have a general type as a generic parameter
  either as an explicit parameter or as implicit in the iterators;

\item the ability to associate an assignment operation with a type
  parameter, either implicitly as a property of all types, or
  as a property of the type definition, or as an explicit parameter to
  the generic with a defined interface;

\item the ability to associate a comparison operation with a
  generic type parameter, either implicitly
  as a property of the type definition, or as an explicit parameter to
  the generic with a defined interface; and

\item the ability to take two iterators as instantiation
  parameters.

\end{itemize}

\newusecase{Permutation}{0-0-0}
It is sometimes useful to perform a permutation of a collection or
verify that one collection is considered a permutation of the other
under a binary logical predicate.  As an example we will consider the
inquiry function {\rm IS\_PERMUTATION} that returns true if the first
collection is considered a permutation of the second. To do this it
first iterates over each collection generating a sorted collection and
then compares each sorted collection element by element to see if the
sorted collections are identical.

A generic {\rm IS\_PERMUTATION} procedure has the
following requirements:
\begin{itemize}
\item the ability to have a general type as a generic parameter
  either as an explicit parameter or as implicit in the iterators;

\item the ability to associate an assignment operation with a type
  parameter, either implicitly as a property of all types, or
  as a property of the type definition, or as an explicit parameter to
  the generic with a defined interface;

\item the ability to associate a comparison operation with a
  generic type parameter, either implicitly
  as a property of the type definition, or as an explicit parameter to
  the generic with a defined interface; and

\item the ability to take two iterators as instantiation
  parameters.

\end{itemize}

\newusecase{Inner Product}{0-0-0}
It is sometimes useful to perform numerical operations on one or two
collections. The operations can take several forms, e.g., inner
product, accumulate, adjacent difference,  etc. We will use the {\rm
  INNER\_PRODUCT}  as an example.  The {\rm INNER\_PRODUCT} iterates
over two collections simultaneously, takes the ``product'' of
corresponding elements and sums the resulting products.

A generic {\rm INNER\_PRODUCT} procedure has the
following requirements:
\begin{itemize}
\item the ability to have a general type as a generic parameter
  either as an explicit parameter or as implicit in the iterators;

\item the ability to associate an assignment operation with a type
  parameter, either implicitly as a property of all types, or
  as a property of the type definition, or as an explicit parameter to
  the generic with a defined interface;

\item the ability to associate a ``product'' operation with a
  generic type parameter, either implicitly
  as a property of the type definition, or as an explicit parameter to
  the generic with a defined interface; and

\item the ability to associate a ``summation'' operation with a
  generic type parameter, either implicitly
  as a property of the type definition, or as an explicit parameter to
  the generic with a defined interface; and

\item the ability to take two iterators as instantiation
  parameters and synchronize them.

\end{itemize}



\subsection{Generic Containers}

 As scientific models become more complex, an increasing fraction of
   the lines of code are infrastructure as-opposed to direct
   numerical calculation.  Examples of infrastructure include software
   layers that couple independently developed subsystems, frameworks
   for managing distributed parallelism, advanced I/O (e.g.,
   checkpoint/restart via NetCDF), etc.  Generally these
   infrastructure layers evolve as an attempt to avoid code
   duplication as multiple parts of the system require similar
   functionality.

 Software "containers" are abstractions that enable aggregating groups of related entities for convenient and efficient access.   Many categories of software containers have been defined, with each   category generally tailored to a common design issue.  By providing a   "standard" and reliable means to perform commonly occurring   operations, containers can greatly simplify design and   implementation of complex algorithms.

Fortran provides just one category of software container - Array.   Array containers are designed to optimize random access to a  {\em fixed} collection of elements, and generally requires all   contained elements to be of the same dynamic type.  Fortran   provides excellent mechanism for declaring and constructing arrays   of any type as well as for accessing, storing, and modifying array    members (elements) with succinct, tailored syntax: tuples of indices within parens:
   
\begin{lstlisting}[language=Python]
   a(i,j) = x
   x = a(i,j)
   a(i,j) = a(i,j)**2
\end{lstlisting}


 Other commonly used container categories include List, Vector, Map (or Associative Array, Dictionary), Set, Stack, Queue, etc.   Unlike Array containers, these others generally are more dynamic -   growing as necessary when new elements are added to the container.   Here I briefly describe some of the most commonly categories of   containers.

\newusecase{List}{0-0-0} 
Lists are sequentially accessed containers. They are
particularly useful when the data needs to be modified, as they are
O(1) complexity in prepending, appending, inserting, and deleting an
element. They come in several forms of which the most prominent are
singly linked lists (providing inexpensive sequential access in one
direction), doubly linked lists (providing slightly more expensive
sequential access in two directions); and flattened lists (basically
lists of rank one arrays).

All the types of lists have similar requirements:
\begin{itemize}
\item a generic construct with scope equivalent to that of a module
  encompassing types and associated procedure definitions;

\item the ability to have a type as a parameter;

\item the ability to associate an assignment operation with a type, either
  implicitly as a property of all types, implicitly as a property of
  the type definition, or as an explicit parameter to the generic
  with a defined interface;

\item the ability to iterate over the active elements of the structure;

\item instantiation of a generic module with a derived type with the same
  parameters should define the same type, or the same instantiation
  should be capable of defining different entities in different
  contexts; and

\item instantiation of a module with a derived type with different
  parameters should define different types; 
\end{itemize}
While not required, a generic list would benefit from the
following capabilities:
\begin{itemize}

\item the ability to specialize depending on whether an instantiation
  type parameter has a LEN parameter or whether it is to be treated as
  polymorphic; 

\item the ability to modify elements of the structure while iterating
  over it; 

\item instantiation in the same specification part that defines the
  types of the data;

\item a language defined iteration construct that can go in two
  directions;

\item a language defined iteration construct that allows modification
  of the structure; and

\item a way of defining a type so that objects of that type can be
  indexed to access elements of that type.

\end{itemize}

\newusecase{Vector}{0-0-0}
Vectors are somewhat similar to 1-D Arrays in that
they provide efficient random access.  But whereas the size of an
Array is fixed at the time of its construction, a Vector can grow as
elements are appended.  (For those unfamiliar with containers, there
is no magic here.  Internally arrays are reallocated with copies, but
by, say, doubling the size when reallocating, appending elements is of
O(1) complexity on average.) The requirements for this generic
container are very similar to those for Lists.

\newusecase{Set}{0-0-0}
Set containers provide efficient mechanisms for
searching and inserting distinct elements.  One implementation
(ordered sets) uses a balanced binary search tree data structure to
store the elements (keys) providing O(log(n)) cost for searching and
inserting distinct elements. This structure requires the availability
of an ordering operation on the key type. Another implementation
(unordered sets) uses a hash table
to store the elements providing O(1) cost for searching and inserting
distinct elements.  This structure requires the availability of hash
and equality operations on the key type.

A generic Set has the following requirements:
\begin{itemize}
\item a generic construct with scope equivalent to that of a module
  encompassing types and associated procedure definitions;

\item the ability to have a type as a parameter;

\item the ability to associate an assignment operation with a type
  parameter, either implicitly as a property of all types, implicitly
  as a property of the type definition, or as an explicit parameter to
  the generic with a defined interface;

\item the ability to iterate over the active elements of the structure;

\item instantiation of a generic module with a derived type with different
  parameters should define different types; and

\item instantiation of a generic module with a derived type with the same
  parameters should define the same type, or the same instantiation
  should be capable of defining different entities in different
  contexts; and

\item the ability to specialize the code based on the rank of the data of 
  interest, the presence or absence of the pointer attribute, and the
  presence or absence of a LEN parameter; 

\end{itemize}
In addition, the ordered Set has the following requirement:
\begin{itemize}

\item the ability to associate an ordering operation with a type
  parameter, either implicitly as part of the type definition or as an
  explicit parameter to the generic with a defined interface;
\end{itemize}
while the unordered Set has the following requirements:
\begin{itemize}
\item the ability to associate an equality operation with a type
  parameter, either implicitly as part of the type definition or as an
  explicit parameter to the generic with a defined interface; and

\item the ability to associate an arbitrary procedure with a type
  parameter, either implicitly as part of the type definition or as an
  explicit parameter to the generic with a defined interface.
\end{itemize}
While not required a generic Set would benefit from the following
capabilities:
\begin{itemize}

\item instantiation in the same specification part that defines the types
  of the key and data;

\item a language defined iteration construct; and

\item the ability to specialize the code based on the rank of the entities
  with a given instantiation type parameter;
\end{itemize}
While not required a generic Set might benefit from the following
capabilities:
\begin{itemize}
\item the ability to define an interface that is independent as to whether 
  the procedure arguments are polymorphic and whether the procedure is
  type bound; 

\item the ability to specialize the code based on the presence or absence
  of the pointer attribute associated with a given instantiation type
  parameter;

\item the ability to specialize the code based on the presence or absence
  of a LEN or KIND parameter associated with a given instantiation
  type parameter; and

\item the ability to instantiate with scalar parameters of type logical,
  integer, or real.
\end{itemize}

\newusecase{Map}{0-0-0} 
Maps (also called associative arrays or dictionaries) are similar to
Sets, except that the contained elements
are key-value pairs.  Typically the keys are either integers or
strings, but can be any data type that can be either ordered (an
ordered Map) or hashed (an unordered Map).  A simple example would be
a Map whose keys are the names of students in a classroom, and the
values are their grades.  A more relevant example would be a Map that
provides efficient access to a sparse array where the keys are the
indices for the non-zero array elements.

The requirements for generic Maps are similar to those for generic
sets except that, with the addition of the value attribute, they
require the ability to have multiple types as parameters.

\newusecase{Bitset}{0-0-0}
The bitset, also known as the bit string, bit map, bit vector, or bit
array, optimizes memory storage by densely packing binary 
information. As it also reduces memory traffic it has the potential of
improving run time performance. As a result both C++ and Java define
bitsets in their standard libraries. However much of this
functionality would be met by the BITS data type proposed for Fortran
202X.

A generic bitset has one instantiation parameter, the number of bits
in a specific set. This number is treated as a run time invariant and
is fixed at compilation time. This number is mapped onto an array of
integers such that the array is the minimum size necessary to
represent that number of bits.  For these bits the bitset type defines
a number of "unary" operations that involve only one bitset, and
"binary" operations that involve two bitsets. The binary operations
only "make sense" if the two bitsets have the same number of bits. By
treating different numbers of bits as representing different types
this constraint is enforced by the type system. The fact that
parameterized derived types treat "LEN" parameters as runtime resolved
makes them impractical to enforce this type safety.

A generic bitset has the following requirements:
\begin{itemize}
\item a generic construct with scope equivalent to that of a module
  encompassing types and associated procedure definitions;

\item the ability to instantiate with a scalar parameter of type integer;

\item instantiation of a generic module with a derived type with different
  parameter values should define different types; 

\item instantiation of a generic module with a derived type with the same
  parameters should define the same type, or the same instantiation
  should be capable of defining different entities in different
  contexts; and

\item the ability to specialize the code based on the value of an integer
  parameter.
\end{itemize}


\newusecase{Iterators}{0-0-0}
Iterators provide a simple/consistent mechanism to efficiently loop through all of the elements in a container.  For   containers such as Vector and Array, these are relatively simple, but for containers such as Set and Map that are implemented with binary trees, the iterator abstraction is extremely beneficial to the user of the container.  Iterators simultaneously hide complexity and encourage safe coding styles.  In pseudo-code iterator usage is typically something like:

\begin{lstlisting}
   < declare container > C
   < declare container iterator > I
   < declare element > e

   I = begin(C)
   loop while I is not end(C)
       e = get(I)
       < do something with e >
       next(I)
   end loop
\end{lstlisting}    


Concrete use cases:
----------

Atmospheric tracers metadata in the NASA GISS climate model.  Each species (CO2, CH4, NO3, ...) in the atmosphere is associated with a variety of metadata.  These include the molar mass, the radioactive decay rate, diffusivity, a category label (dust, aerosol, etc.),   and so on.  There are O(30) such fields in the model currently.   Most are required to have a value for all tracers, but some fields   are only added for tracers where they are relevant.  The types of   these metadata are LOGICAL, INTEGER, REAL, and all but one are   scalars.  The natural representation of this is a Map container
   where the keys are the property name and the values are the various   items of metadata.

Collection of tracers in NASA GISS climate model.  Different   configurations of the model utilize different subsets of the   available tracers in the source code.  The number ranges from 0 to   over 100 tracers.  Each tracer has a natural name, usually the   chemical formula, but sometimes regular names like 'terpene' or   'dust'.  A natural mechanism to manage data associated with each
   tracer is again a Map container where the keys are the tracer names   and the values are a derived type containing all tracer data.  For   historical/conservative reasons only the metadata dictionaries are   actually managed this way.  Other data are in dynamically allocated
   Arrays (Array containers) whose size can be computed after all of   the metadata has been processed.

Regriding registry in NASA GEOS5 model.  This model must represent   data on varying grids that discretize the Earth's atmosphere.  To   transfer data between grids, largish interpolation tables are
   generated and stored.  Because of the expense of computing them,   they are cached at the time they are first computed.  The mechanism   used to manage this is again a Map where the keys are a pair of
   specifiers associated with the two grids that determine the   regriding.


A new I/O client-server layer in NASA GEOS5 model.   

\begin{enumerate}
\item Each server process manages I/O requests from a varying      number of application processes.  This is managed as a Vector of      clients that grows as needed.

\item To encode/decode messages, a Map container with integer keys is       used to store message prototypes.  This allows each subclass of       message to decode itself and while the top-level processing       only needs to handle the integer label.

\item The server processes accumulate data requests from the client        in a Map where they key is a message ID and the details are in        a data type.

\item NetCDF metadata itself uses multiple containers.  NetCDF       variables are represented as a Map with the variable names as       keys and values are Variable derived type.  The Variable
       derived type itself has a Map where the keys are Attribute       names and the values are Attribute values.  The Variable       derived type also has a Vector of "dims" that relate the
       variable dimensions to a global list for the file.
\end{enumerate}

 Next generation pFUnit unit-testing framework for Fortran.  The  framework maintains a global list of "exceptions" that are thrown by user tests.  This is naturally represented as a vector of entities of derived type.

 New configurable logging utility (pFlogger).  This framework   manages collections of abstract data types: Handlers (abstractions   of Files), Loggers, Filters, and Formatters.  To enable runtime
   configurability each such entity is associated with a name.  Map's   are then used to manage the associations.  E.g., each Logger can   have multiple Handlers.  Loggers and Handlers can both have
   multiple Filters.  These Maps all all polymorphic.  The package   also uses a number of containers (Vector and Map) involving   intrinsic types.

\subsubsection{Containers and Fortran}

Infrastructure layers for complex Fortran applications often
implicitly/unknowingly implement crude versions of containers like
Vectors and Maps.  The Vector pattern arises naturally
when the total number of instances of some collection cannot be
determined via a simple formula or when elements are added
sporadically through different drivers.  Instead the size is
"discovered" through some iterative process.  A common implementation
in the simplest case is a two-pass algorithm.  The passes are nearly
identical except that the first pass just accumulates the number of
elements.  An allocation is then made and the 2nd pass fills in the
allocated structure.

The need for Map containers arises when we need to find entities
through some registry.  E.g., if an ocean component needs the surface
winds from an atmospheric component, an abstract coupler enables this
by allowing both components to use the names 'U-wind' and 'V-wind' to
retrieve the associated arrays.  This avoids hardcoding indices, which
is important when considering independently developed components.

Typical Fortran implementations of these patterns are ad-hoc.  Even
within the same application two different layers may implement the
same pattern in rather different ways.  The layers also often contain
latent bugs that are not exercised in the current component, but
prevent their adoption in other components.


\subsubsection{Obstacles to clean/robust implementation of containers in Fortran}

One large obstacle to development and use of containers in Fortran is
the lack of support for generic coding.  Developers must either
duplicate the logic of a container for each potential type of element,
or must resort to non-Fortran means (preprocessors) to have a single
implementation that is applicable to all cases.  E.g., one may wish to
have Vectors of integers, reals, logicals, and/or various derived
types, leading to many almost identical implementations and all the
associated dangers of long-term maintenance.  With Maps, the problem
just grows combinatorially as one considers variant types of keys and
values.

An obstacle to the development of generic containers in Fortran is the
lack of regularity in its type system. The usage of types is
complicated by variants in the syntax due to the distinctions made
between polymorphic (CLASS)  and non-polymorphic (TYPE) entities,
entities with and without a LEN parameter, and with and without a KIND
parameter. Fortran is also unusual in that it makes array and pointer
attributes independent of type rather than part of the type, and that
size, rank, and shape can be declared or assumed. Unless Fortran's
type system is changed to allow a more general syntax, either the
developer of the container will have to use a large number of
specializations, or he will limit the allowed syntax and the user
will have to use workarounds for those limitations, e.g., wrapper
types for polymorphic types for types with a LEN parameter.

Another obstacle, albeit less severe, is the inability to overload
suitable operators for container accessors.  With Fortran arrays,
parens can be used to access individual elements on both the left and
right hand sides of assignment:

      x = a(i,j,k)
      b(i,j,k) = y


Ideally, Vector and Map containers would have similar mechanism for
accessing and modifying elements.  Note that containers generally
return pointers to contained elements rather than copies of those
elements.


\subsubsection{State of the art}

With aggressive and complicated use of CPP/FPP preprocessing it is
possible to write robust generic Vector and Map containers in Fortran.
I and my colleague Doron Feldman have produced such a package which
was recently released as open source: gFTL.  Another similar package,
FTL, has also been independently released.  (They beat us to the
cooler name.)  These implementations are a nice step forward, and have
been of enormous benefit in the development of new powerful software
layers.  However, these containers are still not as simple to use as
one could hope for number of reasons:

1. A separate Fortran class must be constructed for each contained
   type.  In languages with stronger support for containers, this step
   is performed by the compiler.

2. Even if two separate projects use the same container package (e.g.,
   gFTL), they cannot assume that the other is providing a common
   container and must redeclare for themselves.  One easily ends up
   with many all-but-identical modules for say IntegerVector.

3. Iterators are handled via Fortran's DO WHILE construct.  Users then
   put a call to iter%next() at the end of the loop.  This works well
   in simple cases, but is dangerous in the presence of CYCLE because
   the user must remember to invoke iter%next() there as well.  Other
   languages that have an "increment" component to their loop
   constructs (e.g., C, C++) can handle iterators more robustly by
   putting the iterator increment into the header of the loop.  This
   concern is probably more general than just iterators and should
   perhaps be a paper of its own.

4. As mentioned above, the supported accessors are clunky.  The
   inability to use syntax analogous to that of the tuples of indices
   for arrays is unfortunate.  It is a minor annoyance for references
   on the RHS of an assignment, but much more so for references that
   would otherwise be on the LHS.  E.g., if we have a container for a
   sparse array and want to modify an element:

      CALL sparse%set([i,j], x)  ! supported but inelegant

   compared to

      sparse(i,j) = x  ! not implementable in Fortran 2018

5. Containers of pointers are problematic.  Fortran lacks a robust
   mechanism to order pointers via some COMPARE method.  This prevents
   any standard-conforming implementation of a Set of pointers or a
   Map that has keys that are pointers.  Note that usual trick wrapping the
   pointers in a derived type does not help with this issue.


This is a multi-faceted use case and there are a number of fronts on
which the language could be improved to support containers.  Any of
the following would be of some value even without the others:

1) Generic programming to facilitate user-implementation of
   containers that could be used with arbitrary types.  (Ideally,
   community supported packages would then emerge a la C++ STL).

2) Widen the class of overloaded "punctuation" to allow some form of
   paren/bracket notation for specifying container elements on both
   LHS and RGS of assignments.

3) Make specific types containers first class language entities, a la
   existing arrays.

4) Provide an intrinsic COMPARE that arbitrarily (but consistently)
   orders two pointers with the same declared type. Here there are a
   variety of Map and Vector containers that arise.  In many cases,
   the entities in the container are polymorphic.  I.e. the container
   allows entities that are any type that extends some base type.
   (And in at least one case the container entities are of unlimited
   polymorphic type.)  Here are some specific containers in the
   package that are easier to explain:








\newusecase{Block matrix linear algebra}{0-0-0}
T. Clune
\newusecase{Adaptive mesh refinement}{0-0-0}
???

\newusecase{...}{}


\section{Mini use cases}

Mini use cases are narrow aspects derived from the full use cases in the previous section.   A given use case may provide numerous mini use cases, and it is possible that only a subset of these will receive backing from a majority of the committee and thereby feed into the requirements.


\section{Requirements}
\newrequirement{abc}{
This is a requirement
}

\newrequirement{abcd}{
This is another requirement
}

\section{Rejected Use cases}


\end{document}
